\documentclass[main.tex]{subfiles}
\begin{document}

\section{Подобные матрицы. Изменение матрицы линейного оператора при замене базиса}
\newtheorem*{TransMatrixDef}{Определение}
\begin{TransMatrixDef}
	Пусть A и B - квадратные матрицы одного порядка над полем P. Матрица A - называется \textbf{подобной} матрице B над полем P, если существует невырожденная матрица S над полем P такая, что $A = S^{-1}BS$. При этом матрица S называется матрицей, \textbf{трансформирующей} B в A
\end{TransMatrixDef}

\newtheorem*{TransMatrixProp}{Простейшие свойства линейного оператора}
\begin{TransMatrixProp}

	\hfill
	\begin{enumerate}
		
		\item Бинарное отношение подобия матриц является отношением эквивалентности
		\begin{proof}

			\hfill
			\begin{enumerate}
					
				\item Рефлексивность: Матрица A - подобна сама себе, так как трансформирующей матрицей является единичная матрица - E, то есть $A = E^{-1}AE$.

				\item Симметричность: Пусть $A \sim B$, тогда $A = S^{-1}BS$. Так как S - невырожденная матрица, то матрица $S^{-1}$ - также является невырожденной, значит $B = (S^{-1})^{-1} A S^{-1} = SAS^{-1}$, то есть $B \sim A$.

				\item Транзитивность: Пусть $A \sim B$, $B \sim C$, тогда $A = S_{1}^{-1}BS_{1}$, $B = S_{2}^{-1}CS_{2}$. Получаем $A = S_{1}^{-1}S_{2}^{-1}CS_{2}S_{1} = (S_{1}^{-1}S_{2}^{-1})C(S_{2}S_{1}) = (S_{1}S_{2})^{-1}C(S_{2}S_{1})$. Так как $\det{S_{1}}\neq0$ и $\det{S_{2}}\neq0$, $\det{S_{2}S_{1}} = \det{S_{2}}\det{S_{1}}$, то $\det{S_{2}S_{1}}\neq0$. Получили $A \sim C$.
			\end{enumerate}
		\end{proof}

		\item Определители подобных матриц равны

		\begin{proof}
			Пусть $A \sim B$, то есть $A = S^{-1}BS$, значит \[det{A} = \det{S^{-1}}\det{B}\det{S} = \frac{1}{\det{S^{-1}}}\det{B}\det{S} = \det{B}\].
		\end{proof}
		
		\item Ранги подобных матриц равны
		\begin{proof}
			Пусть $A \sim B$, то есть $A = S^{-1}BS$. Из теоремы - ранг произведения матриц, одна из которых невырождена, равен рангу второй матрицы, значит -  $rank A = rank S^{-1}B$, так как S - невырожденная матрица. Подчеркнём, что $S^{-1}$ - также невырожденная матрица, получаем $rank A = rank S^{-1}B = rank B$.
		\end{proof}

	\end{enumerate}
\end{TransMatrixProp}

\noindentПусть $V$ - векторное пространство над полем $P$, $A$ и $B$ - базисы пространства $V$, $f$ - линейный оператор пространства $V$.

\newtheorem*{TSimilarMatrix}{Теорема}
\begin{TSimilarMatrix}
	Пусть $M^{A}$ и $M^{B}$ - подобные матрицы линейного оператора $f$ в базисах $A$ и $B$. При этом матрица $S$, трансформирующая матрицу $M^{A}$ в $M^{B}$, - является матрицей перехода от базиса $A$ к базису $B$
	\[M^{B} = (S_{A\rightarrow B}) ^ {-1} M^{A}S_{A\rightarrow B} = (S_{B\rightarrow A}) M^{A}S_{A\rightarrow B}\]
\end{TSimilarMatrix}
\begin{proof}

\end{proof}
\end{document}